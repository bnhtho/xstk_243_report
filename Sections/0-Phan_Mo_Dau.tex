\section*{\begin{center}
\Huge{Giới thiệu}
	\end{center}}

Trong thời đại khoa học công nghệ phát triển mạnh mẽ, nhu cầu sử dụng linh kiện và thiết bị điện tử ngày càng gia tăng. Điều này đã đặt ra yêu cầu cao hơn về chất lượng sản phẩm và dịch vụ, buộc các cửa hàng điện tử phải đối mặt với sự cạnh tranh khốc liệt. Để thu hút và giữ chân khách hàng, nhiều chương trình khuyến mãi cùng các chính sách ưu đãi được triển khai liên tục, đáp ứng tốt hơn nhu cầu và nâng cao chất lượng đời sống người tiêu dùng.

Để đạt được những mục tiêu này, các cửa hàng điện tử không ngừng đầu tư vào việc phân tích dữ liệu giao dịch bán lẻ. Đây là một bước quan trọng, giúp khảo sát nhu cầu khách hàng, từ đó xây dựng các chiến lược kinh doanh hiệu quả và tối ưu hóa phương thức tiếp cận thị trường.

Với mong muốn có một góc nhìn tổng quát hơn về hoạt động phân tích trong kinh doanh, chúng em đã chọn đề tài “Thống kê dữ liệu bán lẻ giao dịch của cửa hàng điện tử”. Qua đó, chúng em hy vọng có thể tự mình thử nghiệm và áp dụng các kiến thức về xác suất và thống kê đã được tích lũy trong quá trình học tập.

Trong quá trình thực hiện báo cáo, nếu có bất kỳ thiếu sót nào, nhóm rất mong nhận được sự góp ý từ thầy để hoàn thiện tốt hơn.

 \newpage

 