\section{Thống Kê Suy Diễn}
\subsection*{Kiểm tra phân phối chuẩn}
Trước khi lựa chọn các phương pháp thống kê, nhóm phải kiểm tra dữ liệu \textbf{season}, price\textunderscore{total}, order\textunderscore{total}. có tuân theo bảng phân phối chuẩn hay không với phương pháp \textbf{shapiro}.
\begin{lstlisting}[language=R, caption=Shapiro Test]
# Trích xuất các cột order_price và order_total từ merged_data
get_order_price <- merged_data$order_price
get_order_total <- merged_data$order_total

# Kiểm tra phân phối chuẩn với Shapiro-Wilk cho cột order_total
shapiro_order_total <- shapiro.test(get_order_total)
# Kiểm tra phân phối chuẩn với Shapiro-Wilk cho cột order_price
shapiro_order_price <- shapiro.test(get_order_price)

# In kết quả kiểm tra Shapiro-Wilk cho toàn bộ dữ liệu
print(shapiro_order_price)  # Kết quả kiểm tra cho order_price
print(shapiro_order_total)  # Kết quả kiểm tra cho order_total
# Kiểm tra phân phối chuẩn theo từng mùa
seasons <- c("spring", "summer", "fall", "winter")
# Tạo danh sách để lưu giá trị order_price theo từng mùa
order_price_by_season <- list()
# Lặp qua từng mùa, trích xuất dữ liệu và lưu vào danh sách
for (season in seasons) {
  # Trích xuất dữ liệu của từng mùa bằng subset()
  season_data <- subset(merged_data, season == season)  
  # Lưu giá trị order_price của từng mùa vào danh sách
  order_price_by_season[[season]] <- season_data$order_price
}
# Tạo bảng tóm tắt kết quả kiểm tra Shapiro-Wilk cho từng mùa
shapiro_summary <- data.frame(
  Season = names(order_price_by_season),  # Tên các mùa
  P_value = sapply(order_price_by_season, function(order_price) {
    # Tính p-value từ kiểm tra Shapiro-Wilk cho từng mùa
    shapiro.test(order_price)$p.value
  })
)
# Đánh giá phân phối chuẩn: nếu p-value > 0.05 thì phân phối chuẩn
shapiro_summary$Normal_Distribution <- ifelse(shapiro_summary$P_value > 0.05, "True",  # Phân phối chuẩn
"False") # Không phải phân phối chuẩn
# In bảng tóm tắt kết quả kiểm tra Shapiro-Wilk theo mùa
print(shapiro_summary)

\end{lstlisting}
\begin{table}[H]
\centering
\begin{tabular}{|l|r|c|} % Cột: l (trái), r (phải), c (giữa)
\hline
\textbf{Mùa} & \textbf{P-value}       & \textbf{Có phải phân phối chuẩn?} \\ \hline
spring          & 7.348016e-18          & Không                        \\ \hline
summer          & 7.348016e-18          & Không                        \\ \hline
fall            & 7.348016e-18          & Không                        \\ \hline
winter          & 7.348016e-18          & Không                        \\ \hline
\end{tabular}
\caption{Kiểm tra phân phối chuẩn của mùa}
\label{tab:normality_results}
\end{table}
\begin{lstlisting}[language=R,caption=Hai cột còn lại]
> print(shapiro_order_price)  # Kết quả kiểm tra cho order_price
	Shapiro-Wilk normality test
data:  get_order_price
W = 0.95034, p-value < 2.2e-16
> print(shapiro_order_total)  # Kết quả kiểm tra cho order_total
	Shapiro-Wilk normality test
data:  get_order_total
W = 0.94557, p-value < 2.2e-16

% Giới thiệu

\end{lstlisting}
\begin{boxH}
    \textbf{Nhận xét:} Từ những dữ kiện trên, ta có thể kết luận chúng không tuân theo phân phối chuẩn.
\end{boxH}
\subsection{Kiểm định trung bình một mẫu}
\subsubsection{Bài toán thực tế}
Để kiểm định giá trị trung bình một mẫu, nhóm đã đặt giả thuyết rằng doanh thu bán hàng của cửa hàng trong mùa Xuân có bằng với doanh thu kì vọng là 50 triệu đồng hay không? ($\mu$ = $50$)  
Nhóm đưa ra hai giả thuyết như sau:

\textbf{Giả thuyết kiểm định:}
\begin{itemize}
    \item \( H_0 \): \( \mu = 50 \) 
    \item \( H_1 \): \( \mu > 50 \) 
\end{itemize}

\subsubsection{Mô tả dữ liệu và công thức tính}

\begin{lstlisting}[language=R, caption=Tính các thông số]
spring_data <- merged_data %>%
  filter(season == "spring")
## Xuất ra trung bình mẫu, sd , muy
spring_stats <- spring_data %>%
  summarise(
    mean = mean(order_total),
    sd = sd(order_total),
    n = n()
  )  
\end{lstlisting}

\begin{table}[h!]
\centering
\begin{tabular}{|l|r|}
\hline
\textbf{Thông số} & \textbf{Giá trị} \\
\hline
Trung bình mẫu \( \bar{X} \) & 12,320 \\
Giá trị kỳ vọng \( \mu_0 \) & 50 \\
Độ lệch chuẩn mẫu \( s \) & 7,757 \\
Số quan sát \( n \) & 266 (vì df = 265) \\
\hline
\end{tabular}
\caption{Các thông số thống kê dùng trong bài toán kiểm định một mẫu}
\label{tab:thongso}
\end{table}

Áp dụng công thức lý thuyết cho trường hợp phân phối tuỳ ý, mẫu $\geq$ $30$ ,  ta có công thức tính giá trị kiểm định như sau:
\begin{boxH}
\[
Z_{qs} = \frac{\bar{X} - \mu_0}{\dfrac{s}{\sqrt{n}}}
\]

\textbf{Thay số vào:}

\[
Z = \frac{12320 - 50}{\dfrac{7757}{\sqrt{266}}}
= \frac{12270}{474.89} \approx 25.83
\]
\end{boxH}

\subsubsection{Xử lý trên phần mềm RStudio}
Xử lý trên phần mềm RStudio, nhóm sử dụng hàm \texttt{z.test} để tính giá trị kiểm định và p-value. Cụ thể, nhóm sử dụng hàm này với các tham số như sau:

\begin{lstlisting}[language=R, caption=Kiểm định một mẫu]
z.test(x = spring_data$order_total,
       mu = 50,
       sigma.x = sd(spring_data$order_total),
       alternative = "greater")
\end{lstlisting}


\subsubsection{Kết luận}

\begin{lstlisting}[language=R, caption=Kết quả sau khi dùng t-test]
    One-sample z-Test
data:  spring_data$order_price
z = 28.704, p-value < 2.2e-16
alternative hypothesis: true mean is greater than 50
95 percent confidence interval:
 12920.34       NA
sample estimates:
mean of x 
 13702.69 
\end{lstlisting}

\begin{boxH}
    Kết luận: p-value < 0.05, như vậy ta có thể khẳng định , không đủ điều kiện để bác bỏ $H_0$
\end{boxH}


% Ý tưởng hai mẫu (t-test)
\subsection{Kiểm định trung bình hai mẫu}
\subsubsection{Bài toán thực tế}
Sau khi đã xác định được doanh thu của cửa hàng tại mùa xuân không vượt qua giá trị kì vọng, nhóm tiếp tục đem so sánh dữ liệu \textbf{mùa xuân}
so với \textbf{mùa hè} với mức ý nghĩa $\alpha$ = 0.05

Nhóm đặt giả thuyết như sau
\begin{itemize}
    \item \textbf{Giả thuyết không (H\textsubscript{0}):} $\mu_1 = \mu_2$ 
    \item \textbf{Giả thuyết đối (H\textsubscript{1}):} $\mu_1 \ne \mu_2$ 
\end{itemize}
\subsubsection{Tính giá trị kiểm định}
Để tính giá trị kiểm định, nhóm sử dụng phương pháp tách 2 mẫu \textbf{xuân} và \textbf{hè} ra làm 2 và lọc ra thông số độ lệch chuẩn, n (số mẫu) và mean (trung bình mẫu)

\begin{lstlisting}[language=R, caption=Chia nhóm và tính toán chỉ số đặc trưng ]
group_stats <- merged_data %>%
  filter(season %in% c("spring", "summer")) %>%
  group_by(season) %>%
  summarise(
    mean = mean(order_total),
    sd = sd(order_total),
    n = n(),
    .groups = "drop"
  )
print(group_stats)

\end{lstlisting}
\subsubsection{Mô tả dữ liệu và công thức tính}
\begin{table}[H]
\centering
\caption{ Chỉ số bài toán \texttt{order\_total} theo mùa Xuân và Hè}
\begin{tabular}{@{}lccc@{}}
\textbf{Mùa} & \textbf{Trung bình (\textmu)} & \textbf{Độ lệch chuẩn (s)} & \textbf{Số lượng (n)} \\
Xuân (Spring) & 12,320 & 7,757 & 266 \\
Hè (Summer)   & 12,487 & 7,697 & 249 \\
\end{tabular}
\end{table}
Công thức tính cho trường hợp kiểm định 2 mẫu độc lập, có phân phối tuỳ ý và 2 mẫu $\geq$ $30$ như sau:
\[
Z_{qs} = \frac{\overline{X}_1 - \overline{X}_2}
{\sqrt{\frac{s_1^2}{n_1} + \frac{s_2^2}{n_2}}}
\]
\begin{boxH}

Thế số vào:
\[
Z = \frac{12319.55 - 12486.75}{\sqrt{\frac{7757.41^2}{266} + \frac{7697.42^2}{249}}} \approx -0.24542
\]
  
\end{boxH}
\subsubsection{Xử lý trên phần mềm RStudio}
Xử lý trên phần mềm RStudio, nhóm sử dụng hàm \texttt{t.test} để tính giá trị kiểm định và p-value. Cụ thể, nhóm sử dụng hàm này với các tham số như sau:
\begin{lstlisting}[language=R, caption=Các thông số thống kê dùng trong kiểm định trung bình 2 mẫu ]
  t_test_result_two_sample<-t.test(order_total ~ season, 
       data = merged_data %>% filter(season %in% c("spring", "summer")),
       var.equal = FALSE) # giả định phương sai không bằng nhau
print(t_test_result_two_sample)
\end{lstlisting}

\begin{lstlisting}[language=R, caption= Kết quả t-test]
		Welch Two Sample t-test

data:  order_total by season
t = -0.24542, df = 511.26, p-value = 0.8062
alternative hypothesis: true difference in means between group spring and group summer is not equal to 0
95 percent confidence interval:
 -1505.705  1171.298
sample estimates:
mean in group spring mean in group summer 
            12319.55             12486.75 
\end{lstlisting}

\subsubsection{Kết luận}
\begin{boxH}

Với $p_value = 0.8602 > 0.05$ , nhóm không có đủ bằng chứng để bác bỏ giả thuyết $H_0$ tức là không có sự khác biệt trong doanh thu giữa hai mùa.

\end{boxH}
% Anova
\subsection{Anova một yếu tố}

\subsubsection{Bài toán thực tế}
Một cửa hàng điện tử muốn kiểm tra xem \textbf{giá trị đơn hàng trung bình} (\texttt{order\_total}), với cỡ mẫu của mỗi nhóm là $30$, có bị ảnh hưởng bởi \textbf{mùa vụ} (\texttt{season}) hay không.  
Dữ liệu được thu thập từ các đơn hàng, phân loại thành $4$ nhóm theo mùa: \textbf{Winter} (mùa đông), \textbf{Summer} (mùa hè), \textbf{Autumn} (mùa thu) và \textbf{Spring} (mùa xuân).  

Với mức ý nghĩa $\alpha = 5\%$, sử dụng phương ph \textbf{mô hình ANOVA một yếu tố} để kiểm tra xem giá trị đơn hàng trung bình giữa các mùa có sự khác biệt đáng kể hay không.
\subsubsection{Giả thuyết kiểm định}

\begin{itemize}
  \item Các tổng thể phải tuân theo phân phối chuẩn $N(\mu, \sigma^2)$
  \item Số nhóm $k=4$
  \item Phương sai bằng nhau $\sigma_1^2 = \sigma_2^2 = \sigma_3^2 = \sigma_4^2$
  \item Các mẫu quan sát (từ $k$ tổng thể) được lấy độc lập, với $n = 30$ cho mỗi nhóm, tổng số quan sát $N = 120$.
  \item 
\end{itemize}

\subsubsection{Hiện thực phép toán}
Gọi $\mu_i$ là giá trị đơn hàng trung bình của mùa thứ $i$.
\begin{align*}
    H_0 &: \mu_1 = \mu_2 = \mu_3 = \mu_4, \\
    % H_1 &: \exists\ i \neq j \ \text{sao cho} \ \mu_i \neq \mu_j.
    H_1 &: \exists\ \mu_i \neq \mu_j \ \text{sao cho} \ i \neq j.
\end{align*}

\subsubsection{Tính các tham số mẫu}
\begin{table}[H]
\centering
\caption{Trung bình và phương sai mẫu theo mùa}
\begin{tabular}{lcc}
\hline
Mùa & Trung bình mẫu $\overline{x}_i$ & Phương sai mẫu $s_i^2$ \\ \hline
Winter  & $12499.31$ & $56,686,589.91$ \\
Summer  & $12493.37$ & $61,622,987.35$ \\
Autumn  & $12858.15$ & $48,321,908.37$ \\
Spring  & $14594.52$ & $106,019,827.40$ \\ \hline
\end{tabular}
\end{table}
Trung bình tổng thể \textbf{$\overline{x}$} là $1311.33$
\subsubsection{Giá trị trong và giữa các nhóm}

\paragraph{Tổng bình phương giữa nhóm (SSB):}
\[
SSB = 14,303,293.20 + 14,550,442.35 + 3,299,751.675 + 55,465,154.35 = 87,618,641.575
\]
Bình phương trung bình giữa nhóm (MSB):
\[
MSB = \frac{SSB}{k-1} = \frac{87,618,641.575}{3} \approx 29,206,213.86
\]

\paragraph{Tổng bình phương trong nhóm (SSW):}
\[
SSW = 1,643,911,107 + 1,787,066,635 + 1,401,335,343 + 3,074,574,995 = 7,906,888,080
\]
Bình phương trung bình trong nhóm (MSW):
\[
MSW = \frac{SSW}{N-k} = \frac{7,906,888,080}{116} \approx 68,162,828.28
\]

\subsubsection{Giá trị thống kê kiểm định}
\[
F = \frac{MSB}{MSW} = \frac{29,206,213.86}{68,162,828.28} \approx 0.4285
\]

\subsubsection{Miền bác bỏ}
Với $\alpha = 5\%$, $df_1 = k-1 = 3$, $df_2 = N-k = 116$:
\[
F_{0.05; 3; 116} \approx 2.6802
\]
Miền bác bỏ:
\[
RR = (2.6802, +\infty)
\]

\subsubsection{Kết luận}
\begin{boxH}

\textbf{Kết luận:} 
Vì $F \approx 0.4285 < 2.6802$, không bác bỏ $H_0$ nên không có bằng chứng thống kê cho thấy giá trị đơn hàng trung bình giữa các mùa khác biệt đáng kể.

\end{boxH}

\subsection{Xử lý trên phần mềm R}
Sử dụng phương pháp leveneTest trong R để thực hiện
\begin{lstlisting}[language=R, caption=Phép kiểm ANOVA một yếu tố]
df_cleaned <- merged_data %>%
  group_by(season) %>%
  mutate(order_total_cleaned = ifelse(order_total > quantile(order_total, 0.75) + 1.5 * IQR(order_total) | order_total < quantile(order_total, 0.25) - 1.5 * IQR(order_total), NA, order_total)) %>%
  ungroup()
leveneTest(order_total_cleaned ~ season, data = df_cleaned)
anova_model <- aov(order_total_cleaned ~ season, data = df_cleaned)
summary(anova_model)
\end{lstlisting}

\begin{lstlisting}[language=R, caption=Kết quả ANOVA một yếu tố]
Df    Sum Sq   Mean Sq F value Pr(>F)  
season        3 3.576e+08 119214928   2.299 0.0759 .
Residuals   981 5.086e+10  51848700 
  
\end{lstlisting}
Kết luận: Không đủ điều kiện để bác bỏ giả thuyết $H_0$ với $p$-value = 0.0759 > 0.05, tức là không có sự khác biệt đáng kể giữa giá trị đơn hàng trung bình của các mùa.
\subsection{Hồi quy tuyến tính}

\subsubsection{Bài toán thực tế}
Xây dựng mô hình hồi quy tuyến tính với biến phụ thuộc:
\[
Y = \texttt{order\_price}
\]
và hai biến độc lập:
\[
X_1 = \texttt{coupon\_discounts}, \quad X_2 = \texttt{delivery\_charges}.
\]
Mô hình hồi quy bội được viết dưới dạng:
\[
Y = \beta_0 + \beta_1 X_1 + \beta_2 X_2 + \varepsilon_i, \quad i = 1, 2, \ldots, n
\]
trong đó:
\begin{itemize}
    \item $Y$: Giá trị đơn hàng (\texttt{order\_price})
    \item $X_1$: Giảm giá bằng coupon (\texttt{coupon\_discounts})
    \item $X_2$: Phí giao hàng (\texttt{delivery\_charges})
    \item $\varepsilon_i$: Sai số ngẫu nhiên, có kỳ vọng $E(\varepsilon) = 0$, phương sai $\sigma^2$
\end{itemize}

\subsubsection{Cơ sở lý thuyết}
Mô hình hồi quy bội mở rộng mô hình hồi quy đơn bằng cách đưa vào nhiều biến độc lập:
\[
Y = \beta_0 + \beta_1 X_1 + \beta_2 X_2 + \dots + \beta_p X_p + \varepsilon
\]
\begin{itemize}
    \item $\beta_0$: Hệ số chặn
    \item $\beta_j$: Hệ số hồi quy của biến độc lập $X_j$
    \item $\varepsilon$: Sai số ngẫu nhiên
\end{itemize}

\subsubsection{Phương pháp ước lượng}
Sử dụng phương pháp \textbf{Bình phương tối thiểu (OLS)}:
\[
\hat{\beta} = (X^\mathsf{T}X)^{-1} X^\mathsf{T} Y
\]
trong đó:
\begin{itemize}
    \item $X$: Ma trận giá trị các biến độc lập $(n \times (p+1))$
    \item $Y$: Vector giá trị của biến phụ thuộc $(n \times 1)$
    \item $\hat{\beta}$: Vector hệ số hồi quy $(p+1) \times 1$
\end{itemize}

\paragraph{Hệ số xác định $R^2$:}
\[
R^2 = 1 - \frac{SS_\text{res}}{SS_\text{tot}}
\]
với:
\[
SS_\text{res} = \sum_{i=1}^n (y_i - \hat{y}_i)^2, \quad SS_\text{tot} = \sum_{i=1}^n (y_i - \overline{y})^2
\]

\subsubsection{Kết quả mô hình OLS}
Mô hình trong R:
\begin{verbatim}
model <- lm(order_price ~ coupon_discount + delivery_charges, data = df)
summary(model)
\end{verbatim}

\paragraph{Kiểm tra đa cộng tuyến:}
\begin{verbatim}
vif(model)
# coupon_discount: 1.000546
# delivery_charges: 1.000546
\end{verbatim}
$\Rightarrow$ VIF xấp xỉ 1 cho cả hai biến, không xảy ra đa cộng tuyến.

\paragraph{Hệ số ước lượng:}
\[
\hat{\beta}_0 = 13297.85, \quad \hat{\beta}_1 = 143.66, \quad \hat{\beta}_2 = 59.37
\]
Ý nghĩa:
\begin{itemize}
    \item Khi $X_1$ tăng 1 đơn vị, $Y$ tăng trung bình $143.66$
    \item Khi $X_2$ tăng 1 đơn vị, $Y$ tăng trung bình $59.37$
\end{itemize}

\paragraph{Kiểm định $t$:}
\begin{itemize}
    \item $X_1$: $p = 0.527 > 0.05$ $\Rightarrow$ không đủ bằng chứng biến có ý nghĩa
    \item $X_2$: $p = 0.662 > 0.05$ $\Rightarrow$ không đủ bằng chứng biến có ý nghĩa
\end{itemize}

\paragraph{Kiểm định $F$:}
$p$-value lớn hơn 0.05, $R^2 \approx 0$ $\Rightarrow$ mô hình không có ý nghĩa thống kê, biến phụ thuộc quan hệ yếu với các biến độc lập.
% Update 5/12/2024
\vfil\penalty-200\vfilneg
\section{Thảo luận và Mở rộng}

\subsection{Nhận xét dữ liệu}
\begin{itemize}
    \item  Trong mô hình tổng quan, có những cột dữ liệu không tuân theo phân phối chuẩn như \textbf{order\_total} và \textbf{order\_price}. Điều này có thể do sự hiện diện của các giá trị ngoại lai hoặc phân phối không đồng nhất.
    \item Phương sai không đồng nhất , có thể do sự biến động lớn trong giá trị đơn hàng giữa các mùa. 
    \item Có những cột phụ thuộc với nhau, ví dụ \textbf{order\_price} ảnh hưởng bởi \textbf{coupon\_discount}.

\end{itemize}

\subsection{Tính bao quát}

Nhóm đã phân tích các yếu tố ảnh những ảnh hưởng đến sự thay đổi của dữ liệu, cũng như phân tích và kiểm định trong từng trường hợp như sau:
Các kiểm định được chọn phù hợp với mục tiêu và đặc điểm dữ liệu (không chuẩn, không đồng nhất phương sai, có cặp). Ngoài ra, nhóm cũng mở rộng bằng kiểm định hậu nghiệm (Dunn Test) để phân tích sâu hơn sau Kruskal-Wallis. Do đó, phân tích được đánh giá là có tính bao quát tốt và hợp lý về mặt phương pháp.
Cụ thể các phương pháp đã được nhóm sử dụng như sau: 
\begin{itemize}
    \item Phân tích 1 mẫu với \textbf{z-test}
    \item Phân tích 2 mẫu với \textbf{t-test}
    \item Phân tích Annova \textbf bằng {Krustal-Wallis}
    \item Phân tích hồi quy tuyến tính
\end{itemize}
\subsection{Điểm mạnh và điểm yếu của các phương pháp đã sử dụng}

\begin{table}[H]
\centering
\caption{Tổng hợp ưu điểm và nhược điểm của các phương pháp kiểm định}
\label{tab:summary_tests}
\renewcommand{\arraystretch}{1} % Giãn dòng trong bảng
\begin{tabular}{|p{3.2cm}|p{5.5cm}|p{5.5cm}|}
    
\hline
\textbf{Phương pháp} & \textbf{Ưu điểm} & \textbf{Nhược điểm} \\ \hline
% 1 mẫu
\textbf{Z-test một mẫu} &
\begin{itemize}[leftmargin=*, topsep=0pt, partopsep=0pt, parsep=0pt, itemsep=0pt]
\item Dễ hiểu và triển khai.
    \item Hiệu quả với cỡ mẫu lớn (n > 30).
    \item Không yêu cầu phân phối chuẩn nhờ định lý giới hạn trung tâm.
\end{itemize} &
\begin{itemize}[leftmargin=*, topsep=0pt, partopsep=0pt, parsep=0pt, itemsep=0pt]
    \item Giả định dữ liệu độc lập, mẫu đại diện.
    \item Không phù hợp với cỡ mẫu nhỏ.
\end{itemize} \\ \hline
% 2 mẫu
\textbf{T-test hai mẫu (Welch)} &
\begin{itemize}[leftmargin=*, topsep=0pt, partopsep=0pt, parsep=0pt, itemsep=0pt]
    \item So sánh hai nhóm độc lập.
    \item Không yêu cầu phương sai bằng nhau.
\end{itemize} &
\begin{itemize}[leftmargin=*, topsep=0pt, partopsep=0pt, parsep=0pt, itemsep=0pt]
    \item Nhạy cảm với ngoại lai nếu dữ liệu không chuẩn.
    \item Cần số lượng mẫu đủ lớn nếu phân phối không chuẩn.
\end{itemize} \\ \hline
% Annova
\textbf{Anova 1 yếu tố}&
\begin{itemize}[leftmargin=*, topsep=0pt, partopsep=0pt, parsep=0pt, itemsep=0pt]
    \item  Dễ hiểu và thực hiện
    \item Có khả năng mở rộng hay kết hợp với các kiểm định bổ sung như Tukey's HSD để xác định cụ thể nhóm nào khác biệt.
\end{itemize} &
\begin{itemize}[leftmargin=*, topsep=0pt, partopsep=0pt, parsep=0pt, itemsep=0pt]
    \item Yêu cầu tổng thể có phân phối chuẩn.
    \item Không chỉ ra cặp nhóm nào khác biệt.
    \item Cần kiểm định hậu nghiệm (như Dunn Test).
\end{itemize} \\ \hline

% Hồi quy tuyến tính
\textbf{Phương pháp OLS trong hồi quy tuyến tính} &
\begin{itemize}[leftmargin=*, topsep=0pt, partopsep=0pt, parsep=0pt, itemsep=0pt]
    \item Dễ hiểu, tính toán nhanh, nhiều công cụ kiểm định, nền tảng lý thuyết vững
\end{itemize} &
\begin{itemize}[leftmargin=*, topsep=0pt, partopsep =0pt, parsep=0pt, itemsep=0pt]
    \item Nhạy cảm với ngoại lai, cần giả định chặt chẽ, không xử lý tốt dữ liệu phi tuyến, đa cộng tuyến làm mất ổn định
\end{itemize} \\ \hline 

\end{tabular}
\end{table}

\subsection{Mối quan hệ giữa mục tiêu và phương pháp được chọn}

Mục tiêu chính của nghiên cứu không phải là dự đoán giá trị, mà là kiểm tra xem các yếu tố như mùa vụ, đặc điểm sản phẩm hay hành vi mua hàng có tạo ra sự khác biệt có ý nghĩa thống kê về giá trị đơn hàng hay không. 

Nhóm lựa chọn các phương pháp sau:
\begin{enumerate}
    \item \textbf{Z-test một mẫu}: kiểm tra sự khác biệt giữa trung bình mẫu và giá trị giả thuyết.
    \item \textbf{T-test hai mẫu (Welch)}: so sánh trung bình của hai nhóm độc lập khi phương sai không đồng nhất.
    \item \textbf{ANOVA một yếu tố}: kiểm tra sự khác biệt trung bình giữa nhiều nhóm; nếu có ý nghĩa thống kê, mở rộng thêm \textbf{Dunn test} để xác định cặp nhóm khác biệt.
    \item \textbf{Hồi quy tuyến tính}: phân tích tác động của biến độc lập lên biến phụ thuộc và đánh giá mức độ giải thích của mô hình.
\end{enumerate}

Việc lựa chọn trên đảm bảo phù hợp với đặc điểm dữ liệu, đáp ứng mục tiêu so sánh sự khác biệt và phân tích mối quan hệ, thay vì thuần túy dự đoán.
\subsection{Mở rộng với phương pháp Kruskal-Wallis}
\begin{boxH}
    Nguyên lý của phương pháp Kruskal-Wallis
\begin{itemize}
    \item Sắp xếp tất cả các quan sát từ nhỏ đến lớn và gán thứ hạng (rank).
    \item Tính tổng thứ hạng cho từng nhóm.
    \item So sánh sự khác biệt tổng thứ hạng giữa các nhóm để đánh giá liệu các nhóm có xuất phát từ cùng một phân phối hay không.
\end{itemize}
\end{boxH}

Lí do nhóm chọn \textbf{Kruskal-Wallis} là vì:
\begin{itemize}
    \item Dữ liệu không tuân theo phân phối chuẩn và có phương sai không đồng nhất.
    \item Kruskal-Wallis là phương pháp phi tham số, không yêu cầu giả định phân phối chuẩn.
    \item Phù hợp để so sánh nhiều nhóm độc lập, đặc biệt khi dữ liệu không đáp ứng giả định của ANOVA.
\end{itemize}
\subsection{Thực hiện trong R}
\begin{lstlisting}[language=R, caption={Phép kiểm Kruskal–Wallis trong R}]
kruskal_result <- kruskal.test(order_price ~ season, data = merged_data)
# Hiển thị kết quả
print(kruskal_result)
\end{lstlisting}

\begin{lstlisting}[language=R, caption={Kết quả kiểm định Kruskal-Wallis trong R}]
	Kruskal-Wallis rank sum test
data:  order_price by season
Kruskal-Wallis chi-squared = 2.2004, df = 3, p-value = 0.5319
\end{lstlisting}

\textbf{Kết quả:} Kiểm định Kruskal–Wallis cho thấy $\chi^2 = 2.2004$, df $= 3$, p-value $= 0.5319$. 
Vì p-value lớn hơn mức ý nghĩa 0.05, ta không bác bỏ giả thuyết gốc. 
Điều này nghĩa là chưa có bằng chứng thống kê để kết luận giá trị \texttt{order\_price} khác biệt giữa các mùa.

\section{Nguồn Dữ Liệu}
\begin{itemize}
% \href{http://www.overleaf.com}{Something Linky} 
    \item Dữ liệu mẫu: \href{https://www.kaggle.com/datasets/muhammadshahrayar/transactional-retail-dataset-of-electronics-store}{https://www.kaggle.com/datasets/muhammadshahrayar/transactional-retail-dataset-of-electronics-store}
    \item Link R của bài tập lớn: 
    \href{https://github.com/bnhtho/btl_xtsk_241/blob/main/2333017_Assigment.R}{https://github.com/bnhtho/btl\textunderscore{xtsk}\textunderscore{241}/blob/main/2333017\textunderscore{Assigment}.R}
    
\end{itemize}
\section{Tài liệu Tham Khảo}
% Tham khảo
\begin{thebibliography}{99}
\bibitem{anderson} 
Anderson, D. R., Sweeney, D. J., \& Williams, T. A. (2016). 
\textit{Statistics for Business and Economics} (11th ed.). Cengage Learning Vietnam Company Limited.

\bibitem{peterdaagard}
Peter Dalgaard.
\textit{Introductory Statistics with R} (8th ed)
\bibitem{nguyendinhhuy}
Nguyễn Đình Huy.
\textit{Giáo trình xác suất và thống kê(2023)}, Nxb. Đại học Quốc Gia, Thành phố Hồ Chí Minh.

\end{thebibliography}
