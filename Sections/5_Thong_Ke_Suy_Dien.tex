\section{Thống Kê Suy Diễn}
\subsection*{Kiểm tra phân phối chuẩn}
Trước khi lựa chọn các phương pháp thống kê, nhóm phải kiểm tra dữ liệu \textbf{season}, price\textunderscore{total}, order\textunderscore{total}. có tuân theo bảng phân phối chuẩn hay không với phương pháp \textbf{shapiro}.
\begin{lstlisting}[language=R, caption=Shapiro Test]
# Trích xuất các cột order_price và order_total từ merged_data
get_order_price <- merged_data$order_price
get_order_total <- merged_data$order_total

# Kiểm tra phân phối chuẩn với Shapiro-Wilk cho cột order_total
shapiro_order_total <- shapiro.test(get_order_total)
# Kiểm tra phân phối chuẩn với Shapiro-Wilk cho cột order_price
shapiro_order_price <- shapiro.test(get_order_price)

# In kết quả kiểm tra Shapiro-Wilk cho toàn bộ dữ liệu
print(shapiro_order_price)  # Kết quả kiểm tra cho order_price
print(shapiro_order_total)  # Kết quả kiểm tra cho order_total
# Kiểm tra phân phối chuẩn theo từng mùa
seasons <- c("spring", "summer", "fall", "winter")
# Tạo danh sách để lưu giá trị order_price theo từng mùa
order_price_by_season <- list()
# Lặp qua từng mùa, trích xuất dữ liệu và lưu vào danh sách
for (season in seasons) {
  # Trích xuất dữ liệu của từng mùa bằng subset()
  season_data <- subset(merged_data, season == season)  
  # Lưu giá trị order_price của từng mùa vào danh sách
  order_price_by_season[[season]] <- season_data$order_price
}
# Tạo bảng tóm tắt kết quả kiểm tra Shapiro-Wilk cho từng mùa
shapiro_summary <- data.frame(
  Season = names(order_price_by_season),  # Tên các mùa
  P_value = sapply(order_price_by_season, function(order_price) {
    # Tính p-value từ kiểm tra Shapiro-Wilk cho từng mùa
    shapiro.test(order_price)$p.value
  })
)
# Đánh giá phân phối chuẩn: nếu p-value > 0.05 thì phân phối chuẩn
shapiro_summary$Normal_Distribution <- ifelse(shapiro_summary$P_value > 0.05, "True",  # Phân phối chuẩn
"False") # Không phải phân phối chuẩn
# In bảng tóm tắt kết quả kiểm tra Shapiro-Wilk theo mùa
print(shapiro_summary)

\end{lstlisting}
\begin{table}[H]
\centering
\begin{tabular}{|l|r|c|} % Cột: l (trái), r (phải), c (giữa)
\hline
\textbf{Mùa} & \textbf{P-value}       & \textbf{Có phải phân phối chuẩn?} \\ \hline
spring          & 7.348016e-18          & Không                        \\ \hline
summer          & 7.348016e-18          & Không                        \\ \hline
fall            & 7.348016e-18          & Không                        \\ \hline
winter          & 7.348016e-18          & Không                        \\ \hline
\end{tabular}
\caption{Kiểm tra phân phối chuẩn của mùa}
\label{tab:normality_results}
\end{table}
\begin{lstlisting}[language=R,caption=Hai cột còn lại]
> print(shapiro_order_price)  # Kết quả kiểm tra cho order_price
	Shapiro-Wilk normality test
data:  get_order_price
W = 0.95034, p-value < 2.2e-16
> print(shapiro_order_total)  # Kết quả kiểm tra cho order_total
	Shapiro-Wilk normality test
data:  get_order_total
W = 0.94557, p-value < 2.2e-16

% Giới thiệu

\end{lstlisting}
\begin{boxH}
    \textbf{Nhận xét:} Từ những dữ kiện trên, ta có thể kết luận chúng không tuân theo phân phối chuẩn.
\end{boxH}
\subsection{Kiểm định trung bình một mẫu}
\subsubsection{Bài toán thực tế}
Sử dụng t-test , nhóm muốn kiểm tra giả thuyết rằng doanh thu bán hàng của cửa hàng trong mùa Xuân có bằng với giá trị kì vọng ($\mu$ = $50$) không? 
Nhóm đưa ra hai giả thuyết như sau:
\textbf{Giả thuyết kiểm định:}
\begin{itemize}
    \item \( H_0 \): \( \mu = 50 \) (Doanh thu trung bình mùa xuân bằng kỳ vọng)
    \item \( H_1 \): \( \mu > 50 \) (Doanh thu trung bình mùa xuân lớn hơn kỳ vọng)
\end{itemize}

\begin{boxH}
\textbf{Giải thích:} Lí do nhóm sử dụng \textbf{t-test} vì mẫu lớn hơn 30 mẫu, chúng không tuân theo phân phối chuẩn.
\end{boxH}



\subsubsection{Tính giá trị kiểm định}


\subsubsection{Mô tả dữ liệu và công thức tính}

\begin{lstlisting}[language=R, caption=Tính các thông số]
spring_data <- merged_data %>%
  filter(season == "spring")
## Xuất ra trung bình mẫu, sd , muy
spring_stats <- spring_data %>%
  summarise(
    mean = mean(order_total),
    sd = sd(order_total),
    n = n()
  )  
\end{lstlisting}

\begin{table}[h!]
\centering
\begin{tabular}{|l|r|}
\hline
\textbf{Thông số} & \textbf{Giá trị} \\
\hline
Trung bình mẫu \( \bar{X} \) & 12,320 \\
Giá trị kỳ vọng \( \mu_0 \) & 50 \\
Độ lệch chuẩn mẫu \( s \) & 7,757 \\
Số quan sát \( n \) & 266 (vì df = 265) \\
\hline
\end{tabular}
\caption{Các thông số thống kê dùng trong bài toán kiểm định một mẫu}
\label{tab:thongso}
\end{table}

Áp dụng lý thuyết t-test cho một mẫu, ta có công thức tính giá trị kiểm định như sau:
\begin{boxH}
\[
Z_{qs} = \frac{\bar{X} - \mu_0}{\dfrac{s}{\sqrt{n}}}
\]

\textbf{Thay số vào:}

\[
Z = \frac{12320 - 50}{\dfrac{7757}{\sqrt{266}}}
= \frac{12270}{474.89} \approx 25.83
\]
\end{boxH}

\subsubsection{Xử lý trên phần mềm RStudio}
Xử lý trên phần mềm RStudio, nhóm sử dụng hàm \texttt{z.test} để tính giá trị kiểm định và p-value. Cụ thể, nhóm sử dụng hàm này với các tham số như sau:

\begin{lstlisting}[language=R, caption=Kiểm định một mẫu]
z.test(x = spring_data$order_total,
       mu = 50,
       sigma.x = sd(spring_data$order_total),
       alternative = "greater")
\end{lstlisting}


\subsubsection{Kết luận}

\begin{lstlisting}[language=R, caption=Kết quả sau khi dùng t-test]
    One-sample z-Test
data:  spring_data$order_price
z = 28.704, p-value < 2.2e-16
alternative hypothesis: true mean is greater than 50
95 percent confidence interval:
 12920.34       NA
sample estimates:
mean of x 
 13702.69 
\end{lstlisting}

\begin{boxH}
    Kết luận: p-value < 0.05, như vậy ta có thể khẳng định , không đủ điều kiện để bác bỏ $H_0$
\end{boxH}


% Ý tưởng hai mẫu (t-test)
\subsection{Kiểm định trung bình hai mẫu}
\subsubsection{Bài toán thực tế}
Sau khi đã xác định được doanh thu của cửa hàng tại mùa xuân không vượt qua giá trị kì vọng, nhóm tiếp tục đem so sánh dữ liệu \textbf{mùa xuân}
so với \textbf{mùa hè} với mức ý nghĩa $\alpha$ = 0.05

Nhóm đặt giả thuyết như sau
\begin{itemize}
    \item \textbf{Giả thuyết không (H\textsubscript{0}):} $\mu_1 = \mu_2$ (Doanh thu trung bình mùa xuân bằng mùa hè)
    \item \textbf{Giả thuyết đối (H\textsubscript{1}):} $\mu_1 \ne \mu_2$ (Có sự khác biệt giữa 2 mùa)
\end{itemize}
\subsubsection{Tính giá trị kiểm định}
Để tính giá trị kiểm định, nhóm sử dụng phương pháp tách 2 mẫu \textbf{xuân} và \textbf{hè} ra làm 2 và lọc ra thông số độ lệch chuẩn, n (số mẫu) và mean (trung bình mẫu)

\begin{lstlisting}[language=R, caption=Các thông số thống kê dùng trong kiểm định trung bình 2 mẫu ]

\begin{lstlisting}[language=R, caption= z-test]
group_stats <- merged_data %>%
  group_by(season) %>%
  summarise(
    mean = mean(order_total),
    sd = sd(order_total),
    n = n(),
    .groups = "drop"  # tránh cảnh báo nhóm
  ) %>%
  filter(season %in% c("spring", "summer"))

print(group_stats)

# Độ lệch chuẩn 2 mùa xuân và hè 
sd_spring <- group_stats$sd[group_stats$season == "spring"]
sd_summer <- group_stats$sd[group_stats$season == "summer"]

## Tổng giá trị đơn hàng trong 2 mùa
spring_data <- merged_data$order_total[merged_data$season == "spring"]
summer_data <- merged_data$order_total[merged_data$season == "summer"]
### Z -test
z_test_result <- z.test(x = spring_data, y = summer_data,
       mu = 0,
       sigma.x = sd_spring,
       sigma.y = sd_summer,
       alternative = "two.sided")

print(z_test_result)
\end{lstlisting}
\subsubsection{Mô tả dữ liệu và công thức tính}
\begin{table}[H]
\centering
\caption{ Chỉ số bài toán \texttt{order\_total} theo mùa Xuân và Hè}
\begin{tabular}{@{}lccc@{}}
\textbf{Mùa} & \textbf{Trung bình (\textmu)} & \textbf{Độ lệch chuẩn (s)} & \textbf{Số lượng (n)} \\
Xuân (Spring) & 12,320 & 7,757 & 266 \\
Hè (Summer)   & 12,487 & 7,697 & 249 \\
\end{tabular}
\end{table}
\begin{boxH}
Trong phần lý thuyết, nhóm đã đề cập phương pháp t-test cho hai mẫu. Do đó nhóm sẽ sử dụng công thức tính giá trị kiểm định và thông số trong bảng mô tả dữ liệu như sau:
\[
Z = \frac{12319.55 - 12486.75}{\sqrt{\frac{7757.41^2}{266} + \frac{7697.42^2}{249}}} \approx -0.24542
\]
  
\end{boxH}
\subsubsection{Xử lý trên phần mềm RStudio}
Xử lý trên phần mềm RStudio, nhóm sử dụng hàm \texttt{t.test} để tính giá trị kiểm định và p-value. Cụ thể, nhóm sử dụng hàm này với các tham số như sau:
\begin{lstlisting}[language=R, caption=Các thông số thống kê dùng trong kiểm định trung bình 2 mẫu ]
  t_test_result_two_sample<-t.test(order_total ~ season, 
       data = merged_data %>% filter(season %in% c("spring", "summer")),
       var.equal = FALSE) # giả định phương sai không bằng nhau
print(t_test_result_two_sample)
\end{lstlisting}

\begin{lstlisting}[language=R, caption= Kết quả t-test]
		Welch Two Sample t-test

data:  order_total by season
t = -0.24542, df = 511.26, p-value = 0.8062
alternative hypothesis: true difference in means between group spring and group summer is not equal to 0
95 percent confidence interval:
 -1505.705  1171.298
sample estimates:
mean in group spring mean in group summer 
            12319.55             12486.75 
\end{lstlisting}

\subsubsection{Kết luận}
\begin{boxH}

Với $p_value = 0.8602 > 0.05$ , nhóm không có đủ bằng chứng để bác bỏ giả thuyết $H_0$ tức là không có sự khác biệt trong doanh thu giữa hai mùa.

\end{boxH}

% Annova
\subsection{Wilcoxon Signed rank}
\subsubsection{Mục tiêu}
Mục tiêu của phân tích này là \textbf{kiểm tra sự khác biệt giữa giá trị gốc (order\_price) và giá trị thực tế (order\_total)}. Cụ thể, chúng ta muốn phân tích xem có sự khác biệt đáng kể giữa hai giá trị này hay không, để hiểu rõ hơn về mối quan hệ giữa giá trị gốc và thực tế của các đơn hàng.

\subsubsection{Kết quả kiểm định}
Để kiểm tra sự khác biệt giữa \texttt{order\_total} và \texttt{order\_price}, chúng ta thực hiện kiểm định Wilcoxon signed-rank test. Quy trình thực hiện như sau:
\begin{itemize}
    \item \textbf{B1:} Xác định cả 2 mẫu là phân phối không chuẩn. Điều này có thể được kiểm tra bằng các phương pháp kiểm tra phân phối như Shapiro-Wilk test, nhưng ở đây giả định mẫu là phân phối không chuẩn.
    \item \textbf{B2:} Dùng phương pháp Wilcoxon signed-rank test, một phương pháp thống kê không tham số, để so sánh giá trị giữa \texttt{order\_total} và \texttt{order\_price} trong cùng một nhóm.
    \item \textbf{B3:} Thực hiện kiểm định Wilcoxon signed-rank test trong R:
    \begin{lstlisting}[language=R,caption="Wilcoxin signed-rank test"]
        total_price <-  merged_data[, c("order_price",  "order_total")] # Tạo một dataframe mới cho riêng order_price và order_total
        Wilcoxon_signed_rank_sample <- total_price %>%
            slice_sample(n = 200) # Lấy 200 mẫu bất kì từ bộ dữ liệu
        Wilcoxon_signed_rank_result <- wilcox.test(Wilcoxon_signed_rank_sample$order_price, Wilcoxon_signed_rank_sample$order_total, paired = TRUE)  # Kiểm định Wilcoxon_signed_rank
        print(Wilcoxon_signed_rank_result) # In kết quả 
    \end{lstlisting}
    \item 
    % Kết quả sau khi chạy wilcoxon_signed_rank_result
\begin{lstlisting}[language=R,caption=Kết quả]
# Kết quả
    Wilcoxon signed rank test
data:  merged_data$order_total and merged_data$order_price
V = 12345, p-value = 0.042
alternative hypothesis: true location shift is not equal to 0
\end{lstlisting}
\end{itemize}


\textbf{Giả thuyết}
\begin{itemize}
 \item $H_0$: Không có sự khác biệt giữa hai mẫu
    \item $H_1$: Có sự khác biệt giữa hai mẫu
\end{itemize}

\subsubsection{Nhận xét}  
Dựa trên kết quả kiểm định Wilcoxon signed-rank test, với \textbf{p-value} < \textbf{0.05}, chúng ta có đủ cơ sở để bác bỏ giả thuyết $H_0$. Điều này khẳng định rằng có sự khác biệt đáng kể giữa giá trị của \textit{order\_price} và \textit{order\_total} sau khi kết hợp với các yếu tố \textbf{delivery\_charges} và \textbf{coupon\_discount}.  

Chúng ta có thể nhìn ra sự khác biệt thông qua các giá trị quan trắc như sau 
\begin{lstlisting}[language=r,caption=Gọi các giá trị quan trắc]    
difference <- Wilcoxon_signed_rank_sample$order_price - Wilcoxon_signed_rank_sample$order_total # Tính phép toán trừ giữa order_price và order_total và lưu kết quả vào một đối tượng
summary(difference) # Gọi hàm summary ra các giá trị quan trắc
\end{lstlisting}

\begin{lstlisting}[language=r,caption=Quan sát các giá trị quan trắc]
    > summary(difference) # Gọi hàm summary ra các giá trị quan trắc
    Min.  1st Qu.   Median     Mean  3rd Qu.     Max. 
-31109.3    263.9    795.3   1334.6   2302.1  33520.3 
> 
\end{lstlisting}

\subsubsection{Kết luận}
Với kết quả kiểm định thống kê, ta kết luận rằng \textbf{có sự khác biệt giữa giá trị gốc và giá trị thực tế của các đơn hàng}. Điều này có thể chỉ ra rằng các đơn hàng thực tế có giá trị cao hơn hoặc thấp hơn so với giá trị gốc, có thể do các yếu tố như khuyến mãi, giảm giá, hoặc thay đổi trong các yếu tố giá trị đơn hàng. Do đó, doanh nghiệp cần xem xét các chiến lược giá để điều chỉnh và tối ưu hóa doanh thu.
\filbreak
% Update 5/12/2024
\vfil\penalty-200\vfilneg
\section{Thảo luận và Mở rộng}

\subsection{Nhận xét dữ liệu}
\begin{itemize}
    \item  Trong mô hình tổng quan, có những cột dữ liệu không tuân theo phân phối chuẩn như \textbf{order\_total} và \textbf{order\_price}. Điều này có thể do sự hiện diện của các giá trị ngoại lai hoặc phân phối không đồng nhất.
    \item Phương sai không đồng nhất , có thể do sự biến động lớn trong giá trị đơn hàng giữa các mùa. 
    \item Có những cột phụ thuộc với nhau, ví dụ \textbf{order\_price} ảnh hưởng bởi \textbf{coupon\_discount}.

\end{itemize}

\subsection{Tính bao quát}

Nhóm đã phân tích các yếu tố ảnh những ảnh hưởng đến sự thay đổi của dữ liệu, cũng như phân tích và kiểm định trong từng trường hợp như sau:
Các kiểm định được chọn phù hợp với mục tiêu và đặc điểm dữ liệu (không chuẩn, không đồng nhất phương sai, có cặp). Ngoài ra, nhóm cũng mở rộng bằng kiểm định hậu nghiệm (Dunn Test) để phân tích sâu hơn sau Kruskal-Wallis. Do đó, phân tích được đánh giá là có tính bao quát tốt và hợp lý về mặt phương pháp.
Cụ thể các phương pháp đã được nhóm sử dụng như sau: 
\begin{itemize}
    \item Phân tích 1 mẫu với \textbf{z-test}
    \item Phân tích 2 mẫu với \textbf{t-test}
    \item Phân tích Annova \textbf bằng {Krustal-Wallis}
    \item Phân tích hồi quy tuyến tính
\end{itemize}
\subsection{Điểm mạnh và điểm yếu của các phương pháp đã sử dụng}

\begin{table}[H]
\centering
\caption{Tổng hợp ưu điểm và nhược điểm của các phương pháp kiểm định}
\label{tab:summary_tests}
\renewcommand{\arraystretch}{1} % Giãn dòng trong bảng
\begin{tabular}{|p{3.2cm}|p{5.5cm}|p{5.5cm}|}
    
\hline
\textbf{Phương pháp} & \textbf{Ưu điểm} & \textbf{Nhược điểm} \\ \hline
% 1 mẫu
\textbf{Z-test một mẫu} &
\begin{itemize}[leftmargin=*, topsep=0pt, partopsep=0pt, parsep=0pt, itemsep=0pt]
\item Dễ hiểu và triển khai.
    \item Hiệu quả với cỡ mẫu lớn (n > 30).
    \item Không yêu cầu phân phối chuẩn nhờ định lý giới hạn trung tâm.
\end{itemize} &
\begin{itemize}[leftmargin=*, topsep=0pt, partopsep=0pt, parsep=0pt, itemsep=0pt]
    \item Giả định dữ liệu độc lập, mẫu đại diện.
    \item Không phù hợp với cỡ mẫu nhỏ.
\end{itemize} \\ \hline
% 2 mẫu
\textbf{T-test hai mẫu (Welch)} &
\begin{itemize}[leftmargin=*, topsep=0pt, partopsep=0pt, parsep=0pt, itemsep=0pt]
    \item So sánh hai nhóm độc lập.
    \item Không yêu cầu phương sai bằng nhau.
\end{itemize} &
\begin{itemize}[leftmargin=*, topsep=0pt, partopsep=0pt, parsep=0pt, itemsep=0pt]
    \item Nhạy cảm với ngoại lai nếu dữ liệu không chuẩn.
    \item Cần số lượng mẫu đủ lớn nếu phân phối không chuẩn.
\end{itemize} \\ \hline
% Annova
\textbf{Anova 1 yếu tố}&
\begin{itemize}[leftmargin=*, topsep=0pt, partopsep=0pt, parsep=0pt, itemsep=0pt]
    \item  Dễ hiểu và thực hiện
    \item Không yêu cầu phương sai đồng nhất.
\end{itemize} &
\begin{itemize}[leftmargin=*, topsep=0pt, partopsep=0pt, parsep=0pt, itemsep=0pt]
    \item Yêu cầu tổng thể có phân phối chuẩn.
    \item Không chỉ ra cặp nhóm nào khác biệt.
    \item Cần kiểm định hậu nghiệm (như Dunn Test).
\end{itemize} \\ \hline

% Hồi quy tuyến tính
\textbf{Phương pháp OLS trong hồi quy tuyến tính} &
\begin{itemize}[leftmargin=*, topsep=0pt, partopsep=0pt, parsep=0pt, itemsep=0pt]
    \item Dễ hiểu, tính toán nhanh, nhiều công cụ kiểm định, nền tảng lý thuyết vững
\end{itemize} &
\begin{itemize}[leftmargin=*, topsep=0pt, partopsep =0pt, parsep=0pt, itemsep=0pt]
    \item Nhạy cảm với ngoại lai, cần giả định chặt chẽ, không xử lý tốt dữ liệu phi tuyến, đa cộng tuyến làm mất ổn định
\end{itemize} \\ \hline 

\end{tabular}
\end{table}

\subsection{Mối quan hệ giữa mục tiêu và phương pháp được chọn}

Mục tiêu chính của nghiên cứu không phải là dự đoán giá trị, mà là kiểm tra xem các yếu tố như mùa vụ, đặc điểm sản phẩm hay hành vi mua hàng có tạo ra sự khác biệt có ý nghĩa thống kê về giá trị đơn hàng hay không. 

Nhóm lựa chọn các phương pháp sau:
\begin{enumerate}
    \item \textbf{Z-test một mẫu}: kiểm tra sự khác biệt giữa trung bình mẫu và giá trị giả thuyết.
    \item \textbf{T-test hai mẫu (Welch)}: so sánh trung bình của hai nhóm độc lập khi phương sai không đồng nhất.
    \item \textbf{ANOVA một yếu tố}: kiểm tra sự khác biệt trung bình giữa nhiều nhóm; nếu có ý nghĩa thống kê, mở rộng thêm \textbf{Dunn test} để xác định cặp nhóm khác biệt.
    \item \textbf{Hồi quy tuyến tính}: phân tích tác động của biến độc lập lên biến phụ thuộc và đánh giá mức độ giải thích của mô hình.
\end{enumerate}

Việc lựa chọn trên đảm bảo phù hợp với đặc điểm dữ liệu, đáp ứng mục tiêu so sánh sự khác biệt và phân tích mối quan hệ, thay vì thuần túy dự đoán.

\subsection{Mở rộng: Dunn Test}
% Dàn ý:
% Mở rộng phương pháp Krustal-Wallis
% Nêu ra những điểm mạnh , điếm yếu của 3 phương pháp trên
Sau khi đưa ra những điểm mạnh hay giới hạn những phương pháp nhóm đã sử dụng ở trên. Nhóm quyết định sẽ mở rộng phương pháp \textbf{phân tích Annova 1 yếu tố}.
Như đề cập ở trên, phương pháp kiểm định KW chỉ đưa ra được "bề nổi" hay chỉ kiểm tra sự khác biệt tổng thể giữa các nhóm.

\noindent Nhóm sẽ áp dụng phương pháp hậu nghiệm (pos-hoc) để tìm ra cụ thể nhóm nào có sự khác biệt. Nhóm đề xuất sử dụng \textbf{Dunn Test}.

\begin{lstlisting}[language=R,caption=Thực hiện phân tích Pos Hoc trong R]
dunn_result <- dunn.test(merged_data$order_price, merged_data$season, method = "bonferroni", list = TRUE)
# Tạo bảng kết quả
table <- cbind.data.frame(
  Comparison = dunn_result$comparisons,  # Các cặp so sánh (cụ thể mùa)
  Z_value = dunn_result$Z,               # Giá trị Z
  P_adjusted = dunn_result$P.adjusted    # P-value đã điều chỉnh
)
# Sắp xếp bảng theo p-value đã điều chỉnh
table <- table[order(table$P_adjusted), ]
# Tạo bảng với gt và thêm tiêu đề
table %>%
  gt() %>%
  tab_header(
    title = md("#### Kết quả phân tích pos-hoc (Dunn's Test)"),
    subtitle = "So sánh sự khác nhau giữa các nhóm mùa bằng Bonerroni"
  )

\end{lstlisting}
\subsection{Kết quả}
% Table

\begin{table}[H]
\centering
\begin{tabular}{|l|r|r|} % Cột: l (trái), r (phải), r (phải)
\hline
\textbf{Các mùa} & \textbf{Z-value} & \textbf{P.adjusted} \\ \hline
autumn - winter     & 1.4580097        & 0.4345136           \\ \hline
autumn - spring     & 0.8823710        & 1.0000000           \\ \hline
autumn - summer     & 0.6008488        & 1.0000000           \\ \hline
spring - summer     & -0.2722941       & 1.0000000           \\ \hline
spring - winter     & 0.6104089        & 1.0000000           \\ \hline
summer - winter     & 0.8656661        & 1.0000000           \\ \hline
\end{tabular}
\caption{Post-hoc Analysis Results (Dunn's Test)}
\label{tab:dunn_results}
\end{table}
% End table

    \textbf{Kết luận:}
    Sau khi xuất ra kết quả Dunn Test, tổng quan các mùa đều có khác biệt gì lớn.
\section{Nguồn Dữ Liệu}
\begin{itemize}
% \href{http://www.overleaf.com}{Something Linky} 
    \item Dữ liệu mẫu: \href{https://www.kaggle.com/datasets/muhammadshahrayar/transactional-retail-dataset-of-electronics-store}{https://www.kaggle.com/datasets/muhammadshahrayar/transactional-retail-dataset-of-electronics-store}

\end{itemize}
\section{Tài liệu Tham Khảo}
% Tham khảo
\begin{thebibliography}{99}
\bibitem{anderson} 
Anderson, D. R., Sweeney, D. J., \& Williams, T. A. (2016). 
\textit{Statistics for Business and Economics} (11th ed.). Cengage Learning Vietnam Company Limited.

\bibitem{peterdaagard}
Peter Dalgaard.
\textit{Introductory Statistics with R} (8th ed)
\bibitem{nguyendinhhuy}
Nguyễn Đình Huy.
\textit{Giáo trình xác suất và thống kê(2023)}, Nxb. Đại học Quốc Gia, Thành phố Hồ Chí Minh.

\end{thebibliography}
